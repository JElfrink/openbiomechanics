% Options for packages loaded elsewhere
\PassOptionsToPackage{unicode}{hyperref}
\PassOptionsToPackage{hyphens}{url}
%
\documentclass[
]{article}
\usepackage{amsmath,amssymb}
\usepackage{lmodern}
\usepackage{iftex}
\ifPDFTeX
  \usepackage[T1]{fontenc}
  \usepackage[utf8]{inputenc}
  \usepackage{textcomp} % provide euro and other symbols
\else % if luatex or xetex
  \usepackage{unicode-math}
  \defaultfontfeatures{Scale=MatchLowercase}
  \defaultfontfeatures[\rmfamily]{Ligatures=TeX,Scale=1}
\fi
% Use upquote if available, for straight quotes in verbatim environments
\IfFileExists{upquote.sty}{\usepackage{upquote}}{}
\IfFileExists{microtype.sty}{% use microtype if available
  \usepackage[]{microtype}
  \UseMicrotypeSet[protrusion]{basicmath} % disable protrusion for tt fonts
}{}
\makeatletter
\@ifundefined{KOMAClassName}{% if non-KOMA class
  \IfFileExists{parskip.sty}{%
    \usepackage{parskip}
  }{% else
    \setlength{\parindent}{0pt}
    \setlength{\parskip}{6pt plus 2pt minus 1pt}}
}{% if KOMA class
  \KOMAoptions{parskip=half}}
\makeatother
\usepackage{xcolor}
\usepackage[margin=1in]{geometry}
\usepackage{longtable,booktabs,array}
\usepackage{calc} % for calculating minipage widths
% Correct order of tables after \paragraph or \subparagraph
\usepackage{etoolbox}
\makeatletter
\patchcmd\longtable{\par}{\if@noskipsec\mbox{}\fi\par}{}{}
\makeatother
% Allow footnotes in longtable head/foot
\IfFileExists{footnotehyper.sty}{\usepackage{footnotehyper}}{\usepackage{footnote}}
\makesavenoteenv{longtable}
\usepackage{graphicx}
\makeatletter
\def\maxwidth{\ifdim\Gin@nat@width>\linewidth\linewidth\else\Gin@nat@width\fi}
\def\maxheight{\ifdim\Gin@nat@height>\textheight\textheight\else\Gin@nat@height\fi}
\makeatother
% Scale images if necessary, so that they will not overflow the page
% margins by default, and it is still possible to overwrite the defaults
% using explicit options in \includegraphics[width, height, ...]{}
\setkeys{Gin}{width=\maxwidth,height=\maxheight,keepaspectratio}
% Set default figure placement to htbp
\makeatletter
\def\fps@figure{htbp}
\makeatother
\setlength{\emergencystretch}{3em} % prevent overfull lines
\providecommand{\tightlist}{%
  \setlength{\itemsep}{0pt}\setlength{\parskip}{0pt}}
\setcounter{secnumdepth}{5}
\usepackage{authblk}
\usepackage{array}
\usepackage{color,soul}
\usepackage{setspace}
\usepackage{etoolbox,lineno}
\usepackage{amsmath}
\usepackage{mathtools}
\usepackage{threeparttable}
\usepackage{tabularx}
\usepackage{hhline}
\usepackage{booktabs}
\usepackage{float}
\usepackage{caption}
\usepackage{array,multirow}
\usepackage{hyperref}
% \usepackage[superscript]{cite}
\usepackage{cite}
\usepackage[acronym,toc,nonumberlist]{glossaries}
\usepackage[small]{titlesec}
\usepackage[none]{hyphenat}
\renewcommand*{\contentsname}{Table of Contents}
\renewcommand{\thefootnote}{\roman{footnote}}
\setlength{\parindent}{0.375in}
\author[1,$\dagger$]{\footnotesize Kyle W Wasserberger}
\author[1]{\footnotesize Anthony C Brady}
\author[1]{\footnotesize Kyle J Boddy}
\affil[1]{\footnotesize Research \& Development; Driveline Baseball}
\affil[$\dagger$]{\footnotesize Corresponding author: kyle.wasserberger@drivelinebaseball.com}
\newcolumntype{L}[1]{>{\small\raggedright\let\newline\\\arraybackslash\hspace{-1pt}}p{#1}}
\newcolumntype{C}[1]{>{\small\centering\let\newline\\\arraybackslash\hspace{-1pt}}p{#1}}
\newcolumntype{R}[1]{>{\small\raggedleft\let\newline\\\arraybackslash\hspace{-1pt}}p{#1}}
\hypersetup{
  colorlinks=true
}
\ifLuaTeX
  \usepackage{selnolig}  % disable illegal ligatures
\fi
\IfFileExists{bookmark.sty}{\usepackage{bookmark}}{\usepackage{hyperref}}
\IfFileExists{xurl.sty}{\usepackage{xurl}}{} % add URL line breaks if available
\urlstyle{same} % disable monospaced font for URLs
\hypersetup{
  pdftitle={The OpenBiomechanics Project},
  hidelinks,
  pdfcreator={LaTeX via pandoc}}

\title{The OpenBiomechanics Project}
\usepackage{etoolbox}
\makeatletter
\providecommand{\subtitle}[1]{% add subtitle to \maketitle
  \apptocmd{\@title}{\par {\large #1 \par}}{}{}
}
\makeatother
\subtitle{The Open Source Initiative for Anonymized, Elite-Level Athletic Motion Capture Data}
\date{\vspace{-2.5em}}

\begin{document}
\maketitle

\pagenumbering{gobble}
\begin{center}
Keywords: open access, baseball, pitching
\end{center}

\bigskip
\bigskip
\begin{abstract}
\doublespacing
Interest in quantifying human movement, particularly in elite sport, increases with each passing year. However, analysis of sport biomechanics data has traditionally been restricted to academic laboratories and professional sport organizations. Public sport biomechanics resources would democratize access to human movement data and accelerate progress and innovation for the sport biomechanics field as a whole. In this paper, we introduce \href{www.openbiomechanics.org}{\textbf{\textit{The OpenBiomechanics Project}}}, an initiative started by \href{https://www.drivelinebaseball.com/research/}{\textbf{\textit{Driveline Baseball Research \& Development}}} to provide free, elite-level, research grade motion capture data to the general public for independent individual exploration and analysis. We begin by providing raw and processed data from a sample of 100 baseball pitchers. We then discuss future directions within baseball, expansion to other sports and athletic movements, and outline supporting documentation and additional resources.
\end{abstract}

\newpage

\hypertarget{introduction}{%
\section{Introduction}\label{introduction}}

Open access resources are available in other biomechanics sub disciplines \cite{erdemir2016open}.

\hypertarget{how-to-use}{%
\section{How to Use}\label{how-to-use}}

\ldots{}

\hypertarget{terms-of-use}{%
\subsection{Terms of Use}\label{terms-of-use}}

\hypertarget{naming-conventions}{%
\subsection{Naming Conventions}\label{naming-conventions}}

\ldots{}

\hypertarget{fileshare-and-github-repository}{%
\subsection{Fileshare and GitHub Repository}\label{fileshare-and-github-repository}}

\ldots{}

\hypertarget{citing-and-contributing}{%
\subsection{Citing and Contributing}\label{citing-and-contributing}}

\ldots{}

\hypertarget{discussion-and-future-directions}{%
\section{Discussion and Future Directions}\label{discussion-and-future-directions}}

\ldots{}

\hypertarget{additional-resources}{%
\section{Additional Resources}\label{additional-resources}}

\ldots{}

\newpage
\singlespacing
\addcontentsline{toc}{section}{References}
\bibliographystyle{plain-custom}
\bibliography{openBiomech}

\end{document}
